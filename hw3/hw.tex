\documentclass{abrice}

\title{Math 354: Homework 3}
\author{Anthony Brice}

\usepackage{tikz}
\usetikzlibrary{positioning}
\usetikzlibrary{decorations.pathreplacing}

\begin{document}
\maketitle

\section{Problem 1}
\emph{Show that the solution of $T(n) = T(n - 1) + n$ is $O(n^2)$.}
\medskip

\begin{proof}
  Assume $T(m) \leq cm^2$ for some $c > 0$ and for all positive $m < n$. Then
  $T(n-1) \leq c{(n-1)}^2$. Then
  \begin{align*}
    T(n) &\leq c{(n - 1)}^2 + n \\
    &= cn^2 - 2cn + c + n \\
    &= cn^2 - (2c - 1)n + c \\
    &\leq cn^2 && \text{for } c = 1 \, .
  \end{align*}
\end{proof}

\section{Problem 2}
\emph{Show that $T(n) = 2T(\floor{n/2}) + n$ is $\Omega(n \lg n)$. Conclude that
  the solution is $\Theta(n \lg n)$.}
\medskip

\begin{proof}
  Assume $T(m) \geq c m \lg m$ for some $c > 0$ and for all positive $m <
  n$. Then $T(\floor{n / 2}) \geq c \floor{n / 2} \lg{\floor{n / 2}}$. Then
  \begin{align*}
    T(n) &\geq 2(c \floor{n / 2} \lg(\floor{n / 2})) + n \\
    &\geq cn \lg(n/2) + n \\
    &= cn \lg n - cn \lg 2 + n \\
    &= cn \lg n - cn + n \\
    &\geq cn \lg n && \text{for } c = 1\, .
  \end{align*}

  Since $cn \lg n \leq T(n) \leq cn \lg n$ for $c = 1$,
  $T(n) = \Theta(n \lg n)$.
\end{proof}

\section{Problem 3}
\emph{Show that by making a different inductive hypothesis, we can overcome the
  difficulty with the boundary condition $T(1) = 1$ for the recurrence above
  without adjusting the boundary conditions for the inductive proof.}
\medskip

\begin{proof}
  Assume the stronger hypothesis that $T(m) \leq c m \lg m + bm$ for some
  $b,c > 0$ and $m < n$. Then
  $T(\floor{n/2}) \leq c \floor{n/2} \lg{\floor{n/2}} + bn$. Then
  \begin{align*}
    T(n) &\leq 2(c \floor{n/2} \lg\floor{n/2}) + n \\
    &\leq cn \lg{(n/2)} + bn + n \\
    &= cn \lg n - cn + bn + n \\
    &\leq cn \lg n + bn && \text{for } c \geq 1 \, .
  \end{align*}

  Then when $n = 1$, $T(1) = 1 \leq c \lg 1 + b$ holds for all $b \geq 1$.
\end{proof}

\section{Problem 4}
\emph{Use a recursion tree to determine a good asymptotic upper bound on the
  recurrence $T(n) = 3T(\floor{n/2}) + n$. Use the substitution method to verify
  your answer.}
\medskip

\begin{figure*}
  \centering
  \begin{tikzpicture}[level/.style={sibling distance=40mm/#1}]
    \node (z) {$n$}
      child {node (a) {$n/2$}
        child {node (e) {$n/4$}
          child {node (j) {$n/8$}
            child [dashed] {node (m) {}}
          }
          child {node (k) {$n/8$}
            child [dashed] {node (n) {}}
          }
          child {node (l) {$n/8$}
            child [dashed] {node (o) {}}
          }
        }
        child {node (f) {$n/4$}
          child [dashed] {node (m) {}}
        }
        child {node (g) {$n/4$}
          child [dashed] {node (n) {}}
        }
      }
      child {node (b) {$n/2$}
        child [dashed] {node (h) {}}
      }
      child {node (c) {$n/2$}
        child [dashed] {node (i) {}}
      };

    %\node [below=2.5cm of j.west,anchor=west] {$T(1)$ $T(1)$ $T(1)$ $T(1)$ $T(1)$ $T(1)$ $T(1)$
      %$T(1)$ $T(1)$ $T(1)$ $T(1)$ $T(1)$ $T(1)$ $T(1)$ \ldots $T(1)$};

    \node [below=2.5cm of j.west,anchor=west] (aa) {$T(1)$};
    \node [above=0.5cm of aa] (az) {};
    \draw [dashed] (aa.north) -- (az);

    \node [right=0.5cm of aa] (ba) {$T(1)$};
    \node [above=0.5cm of ba] (bz) {};
    \draw [dashed] (ba.north) -- (bz);

    \node [right=0.5cm of ba] (ca) {$T(1)$};
    \node [above=0.5cm of ca] (cz) {};
    \draw [dashed] (ca.north) -- (cz);

    \node [right=0.5cm of ca] (da) {$T(1)$};
    \node [above=0.5cm of da] (dz) {};
    \draw [dashed] (da.north) -- (dz);

    \node [right=0.5cm of da] (ea) {$T(1)$};
    \node [above=0.5cm of ea] (ez) {};
    \draw [dashed] (ea.north) -- (ez);

    \node [right=0.5cm of ea] (fa) {$T(1)$};
    \node [above=0.5cm of fa] (fz) {};
    \draw [dashed] (fa.north) -- (fz);

    \node [right=0.5cm of fa] (ga) {$T(1)$};
    \node [above=0.5cm of ga] (gz) {};
    \draw [dashed] (ga.north) -- (gz);

    \node [right=0.5cm of ga] (ha) {\ldots};

    \node [right=0.5cm of ha] (ia) {$T(1)$};
    \node [above=0.5cm of ia] (iz) {};
    \draw [dashed] (ia.north) -- (iz);

    \node [left=0.5cm of aa] (zz) {};
    \node [above=6.8cm of zz] (az) {};
    \draw [<->] (zz.south) -- (az) node [midway,fill=white] {$\lg n$};

    \draw [decorate,decoration={brace,amplitude=7pt,mirror},yshift=212pt]
    (aa.south west) -- (ia.south east) node [midway,yshift=-0.6cm]
    {$n^{\lg 3}$};

    \path (z -| i) ++(3cm,0) node {$n$};
    \path (a -| i) ++(3cm,0) node {${3 \over 2} n$};
    \path (e -| i) ++(3cm,0) node {$\left( {3 \over 2} \right)^2 n$};
    \path (j -| i) ++(3cm,0) node {$\left( {3 \over 2}\right)^3 n$};
    \path (o -| i) ++(3cm,0) node {\vdots};
    \path (aa -| i) ++(3cm,0) node {$\Theta(n^{\lg 3})$};
  \end{tikzpicture}
  \caption{A recursion tree of $T(n) = 3T(\floor{n/2}) + n$}
\end{figure*}

\noindent
We have
\begin{align*}
  T(n) &= n + {3 \over 2} n + {\left( {3 \over 2 } \right)}^2 n + {\left( {3
         \over 2 } \right)}^3 n + \cdots + {\left( {3 \over 2 } \right)}^{\lg n -
         1} n + \Theta (n^{\lg 3}) \\
  &= \sum_{i = 0}^{\lg n - 1} {\left( {3 \over 2} \right)}^i n + \Theta(n^{\lg
    3}) \\
  &= {{(3/2)}^{\lg n} - 1 \over (3/2) - 1} n + \Theta(n^{\lg 3}) \, .
\end{align*}

We expect the solution to be at most the number of levels times the cost of each
level, or $O(n \lg n)$.

\begin{proof}
  Assume $T(m) \leq cm \lg m$ for some $c > 0$ and for all positive $m <
  n$. Then $T(\floor{n / 2}) \leq c \floor{n / 2} \lg \floor{n / 2}$. Then
  \begin{align*}
    T(n) &\leq 3 c \floor{n / 2} \lg \floor{n / 2} + n\\
    &\leq 3 c (n / 2) \lg(n / 2) + n \\
    &= {3 \over 2} c n \lg n - {3 \over 2} c n + n \\
    &\leq c n \lg n && \text{for } c \geq {2 \over 3} \, .
  \end{align*}
\end{proof}

\section{Problem 5}
\emph{Use a recursion tree to determine a good asymptotic upper bound on the
  recurrence $T(n) = T(n - 1) + 1$. Use the substitution method to verify your
  answer.}
\medskip

\begin{figure*}
  \centering
  \begin{tikzpicture}[level/.style={sibling distance=40mm/#1}]
    \node (z) {$1$}
      child {node (a) {$n - 1$}
        child {node (b) {$n - 2$}
          child {node (c) {$n - 3$}
            child [dashed] {node (d) {}}}
        }
      };

    \node [below=2cm of c] (e) {$T(1)$};

    \node [left=3cm of z.center] (f) {};
    \node [left=3cm of e.center] (g) {};
    \draw [<->] (f.north) -- (g.south) node [midway,fill=white] {$n$};

    \path (z) ++(3cm,0) node {$1$};
    \path (a) ++(3cm,0) node {$n - 1$};
    \path (b) ++(3cm,0) node {$n - 1(2)$};
    \path (c) ++(3cm,0) node {$n - 1(3)$};
    \path (d) ++(3cm,0) node {\vdots};
    \path (e) ++(3cm,0) node {$\Theta(1)$};
  \end{tikzpicture}
  \caption{A recursion tree of $T(n) = T(n-1) + 1$.}
\end{figure*}

\noindent
We expect the solution to be at most the number of levels ($n$) times the cost
of each level ($n - i$), or $n^2$.

\begin{proof}
  Assume $T(m) \leq cm^2$ for some $c > 0$ and for all positive $m < n$. Then
  $T(n - 1) \leq c {(n - 1)}^2$. Then
  \begin{align*}
    % T(n) &\leq c(n - 1) + 1 \\
    % &= cn - c + 1 \\
    % &\leq cn && \text{for } c = 1 \, .
    T(n) &\leq c{(n - 1)}^2 + 1 \\
    &= cn^2 - 2cn + c + 1 \\
    &\leq cn^2 && \text{for } c = 1 \, .
  \end{align*}
\end{proof}

\section{Problem 6}
\emph{Use the master method to give tight asymptotic bounds for the following
  recurrences.
  \begin{enumerate}[label=(\alph*)]
  \item $T(n) = 2T(n/4) + 1$.
  \item $T(n) = 2T(n/4) + \sqrt{n}$.
  \item $T(n) = 2T(n/4) + n$.
  \item $T(n) = 2T(n/4) + n^2$.
  \end{enumerate}
}
\medskip

\noindent
Note that in all cases $a = 2$ and $b = 4$. Then $\log_4{2} = {1/2}$.
\begin{enumerate}[label=(\alph*)]
\item $f(n) = 1 = O(n^{1/2 - \epsilon})$ when $\epsilon = 1/2$. Then
  $T(n) = \Theta(n/2)$.
\item $f(n) = \sqrt{n} = \Theta(n^{1/2})$. Then $T(n) = \lg n \sqrt{n}$.
\item $f(n) = n = \Omega(n^{1/2 + \epsilon})$ when $\epsilon = 1/2$. Then $T(n)
  = \Theta(n)$.
\item $f(n) = n^2 = \Omega(n^{1/2 + \epsilon})$ when $\epsilon = 3/2$. Then
  $T(n) = \Theta(n^2)$.
\end{enumerate}

\section{Problem 7}
\emph{Use the master method to show that solution to the binary-search
  recurrence $T(n) = T(n/2) + \Theta(1)$ is $T(n) = \Theta(\lg n)$.}
\medskip

\noindent
$a = 1$, $b = 2$, and $f(n) = \Theta(1)$. Since $\Theta(1) = \Theta(n^{\log_2 1})
= \Theta(1)$, case 2 applies. Then $T(n) = \Theta(\lg n)$.

\section{Problem 8}
\emph{Suppose you're choosing among the following three algorithms:
  \begin{enumerate}[label=(\Alph*)]
  \item Solves problems by dividing them into five subdomains of half the size,
    recursively solving each subproblem, and then combining the solutions in
    linear time.
  \item Solves problems of size by $n$ by recursively solving two subproblems of
    size $n - 1$ and then combining the results in constant time.
  \item Solves problems of size $n$ by dividing then into $9$ subproblems of
    size $n/3$, recursively solving each subproblem, and then combining the
    solutions in $O(n^2)$ time.
  \end{enumerate}
  What are the running times of each of these algorithms and which would you
  choose?
}
\medskip

\begin{enumerate}[label=(\Alph*)]
\item $T(n) = 5T(n/2) + n$. Note that case 3 of the master method fails here
  because there exists no $c < 1$ such that $5n/2 \leq cn$. Using a recursion
  tree, we can find that any given problem will be solved in $\lg n$ levels with
  a cost of ${(5/2)}^i n$ each. Therefore, we assign this algorithm an upper
  bound of $O(n \lg n)$.
\item $T(n) = 2T(n - 1) + c$. Using a recursion tree, we find any given problem
  will be solved in $n$ levels with a cost of $2^{n}$ at each level. Therefore
  we assign this algorithm an upper bound of $O(2^n)$.
\item $T(n) = 9T(n/3) + O(n^2)$. Using a recursion tree, we find any given
  problem will be solved in $\log_3 n$ levels with a cost of ${(9/3)}^i O(n^2)$.
  Therefore we assign this algorithm an upper bound of $O(n^2 \log_3 n)$.
\end{enumerate}

Given the above analysis, we would prefer algorithm A over the others.

\section{Problem 9}
\emph{Solve the following recurrences and give a $\Theta$ bound for each.
  \begin{enumerate}[label= (\alph*)]
  \item $T(n) = 7T(n/7) + n$.
  \item $T(n) = 9T(n/3) + n^2$.
  \item $T(n) = 8T(n/2) + n^3$.
  \item $T(n) = T(n - 1) + 2$.
  \item $T(n) = T(\sqrt n) + 1$.
  \end{enumerate}
}
\medskip

\begin{enumerate}[label=(\alph*)]
\item By case 2 of the master method, $T(n) = \Theta(n \lg n)$.
\item By case 2 of the master method, $T(n) = \Theta(n^2 \lg n)$.
\item By case 2 of the master method, $T(n) = \Theta(n^3 \lg n)$.
\item $T(n) = \Theta(n)$.
  \begin{proof}
    We first use the substitution method to show that $T(n) = O(n)$. Assume
    $T(m) \leq cn - d$ for constants $c > 0$ and $d \geq 0$, and for all
    positive $m < n$. Then $T(n - 1) \leq c(n - 1) - d$. Then
    \begin{align*}
      T(n) &\leq c(n - 1) - d + 2 \\
      &= cn - c - d + 2 \\
      &\leq cn - d && \text{for } d \geq 2 \, .
    \end{align*}

    Similarly we use the substitution method to show that $T(n) = \Omega(n)$.
    Assume $T(m) \geq cn$ for some constant $c > 0$ and for all positive
    $m < n$. Then $T(n - 1) \geq c(n - 1)$. Then
    \begin{align*}
      T(n) &\geq c(n - 1) + 2 \\
      &= cn - c + 2 \\
      &\geq cn && \text{for } c < 2 \, .
    \end{align*}

    Therefore since $T(n) = O(n)$ and $T(n) = \Theta(n)$.
  \end{proof}
\item $T(n) = \Theta(\lg(\lg n))$.
  \begin{proof}
    We use the substitution method to show $T(n) = \Theta(\lg(\lg n))$. Assume
    $T(m) = c \lg(\lg m)$ for constant $c > 0$, and for all positive $m <
    n$. Then $T(\sqrt n) = c \lg(\lg \sqrt n)$. Then
    \begin{align*}
      T(n) &= c \lg(\lg \sqrt n) + 1 \\
      &= c \lg\left( {1 \over 2} \lg n \right) + 1 \\
      &= c \lg(\lg n) - c + 1 \\
      &= c \lg(\lg n) && \text{for } c = 1.
    \end{align*}
  \end{proof}
\end{enumerate}

\end{document}
%%% Local Variables:
%%% mode: latex
%%% TeX-master: t
%%% End:
